\documentclass[12pt,letterpaper]{article}

\usepackage[spanish,es-tabla]{babel}
\usepackage{amsmath}
\usepackage[utf8]{inputenc}
\usepackage[T1]{fontenc}
\usepackage{lmodern}
\usepackage{graphicx}
\usepackage{listings}
\usepackage{anysize} 
\usepackage{fancyhdr}
\usepackage{amsmath}
\usepackage{pdfpages}
\usepackage{graphics}
\usepackage{capt-of}
\usepackage[colorlinks=true,plainpages=true,citecolor=blue,linkcolor=blue]{hyperref}

\marginsize{1cm}{1cm}{1cm}{1cm}
\pagestyle{fancy}
\fancyhf{}
\fancyhead[L]{\footnotesize UPIITA-IPN} 
\fancyhead[R]{\footnotesize Lineas de transmisión y antenas} 
\fancyfoot[R]{\footnotesize }
\fancyfoot[C]{\thepage}
\fancyfoot[L]{\footnotesize Formulario} 
\renewcommand{\footrulewidth}{0.4pt}
\renewcommand{\spanishtablename}{Tabla}

\begin{document}

\newpage
\section{Lineas de transmision}

\subsection{Lineas desacopladas}

Voltaje y corriente con origen en la carga en una linea sin perdidas
\begin{equation}
    V(s)=Vo^{+}(e^{j \beta s}+\Gamma(0)e^{-j \beta s})
\end{equation}

\begin{equation}
    I(s)=\frac{Vo^{+}}{z_0}(e^{j \beta s}-\Gamma(0)e^{-j \beta s})
\end{equation}

Impedancia en cualquier punto de la linea cuando se tiene $z_0$ y $s_L$
\begin{equation}
    z(s)=z_{0}(\frac{z_{L}+jz_{0}\tan(\beta s)}{z_{0}+jz_{L}\tan(\beta s)})
\end{equation}

\begin{equation}
    \beta s=\frac{2 \pi s}{\lambda} [rads]
\end{equation}

\begin{equation}
    \beta s=\frac{2 \pi s}{\lambda}
\end{equation}

Relacion de onda estacionaria $ROE$
\begin{equation}
    ROE=\frac{R_{max}}{z_0}
\end{equation}

\begin{equation}
    ROE=\frac{z_0}{R_{min}}
\end{equation}

Impedancia caracteristica de la linea $z_{0}$ cuando se conocen las resistencias
\begin{equation}
    z_{0}=\sqrt{R_{max} R_{min}}
\end{equation}

$\theta_{2}$ Linea completa y $\theta_{1}$ distancia de la resistencia hasta $z_{L}$
\begin{equation}
    \theta=\theta_2-\theta_1
\end{equation}

Cuando se conoce la resistencia minima - $R_{min}$
\begin{equation}
    z_{int}(s)=z_{0}(\frac{R_{min}+jz_{0}\tan(\theta)}{z_{0}+jR_{min}\tan(\theta)})
\end{equation}

\begin{equation}
    z_{int}(s)=z_{0}(\frac{ROE+j\tan(\theta)}{1+jROE\tan(\theta)})
\end{equation}

Cuando se conoce la resistencia maxima - $R_{max}$
\begin{equation}
    z_{int}(s)=z_{0}(\frac{R_{max}+jz_{0}\tan(\theta)}{z_{0}+jR_{max}\tan(\theta)})
\end{equation}

\begin{equation}
    z_{int}(s)=z_{0}(\frac{ROE+j\tan(\theta)}{1+jROE\tan(\theta)})
\end{equation}

\subsection{Impedancia y Admitancia normalizada}
Normalizacion
\begin{equation}
    z(s)=R+jX
\end{equation}

\begin{equation}
    \frac{z(s)}{z_0}=z(s)=\hat{z}=r+jx
\end{equation}

\begin{equation}
    \hat{z}=r+jx=\frac{1+\Gamma(s)}{1-\Gamma(s)}
\end{equation}

Impedancia normalizada para la carta de Smith
\begin{equation}
    \hat{z}=\frac{1+\Gamma(s)e^{-2 \gamma s}}{1-\Gamma(s)e^{-2 \gamma s}}
\end{equation}

Admitancia y coeficiente de reflexion sin perdidas normalizada
\begin{equation}
    \hat{y}=\frac{1}{z}=\frac{1-\Gamma(s)}{1+\Gamma(s)}
\end{equation}

\begin{equation}
    \Gamma(s)=\frac{\hat{z}-1}{\hat{z}+1}=\frac{1-\hat{y}}{1+\hat{y}}
\end{equation}

Admitancia y coeficiente de reflexion con perdidas normalizada
\begin{equation}
    \hat{y}=\frac{1-\Gamma(s)e^{-2 \gamma s}}{1+\Gamma(s)e^{-2 \gamma s}}
\end{equation}

\begin{equation}
    \Gamma(s)e^{-2 \gamma s}=\frac{\hat{z}-1}{\hat{z}+1}
\end{equation}



\end{document}