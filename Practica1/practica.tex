\documentclass[12pt,letterpaper]{article}

\usepackage[spanish,es-tabla,es-nodecimaldot]{babel}
\usepackage{amsmath}
\usepackage[utf8]{inputenc}
\usepackage[T1]{fontenc}
\usepackage{lmodern}
\usepackage{graphicx}
\usepackage{listings}
\usepackage{anysize} 
\usepackage{fancyhdr}
\usepackage{amsmath}
\usepackage{pdfpages}
\usepackage{graphics}
\usepackage{capt-of}
\usepackage{tabularx}
\usepackage[colorlinks=true,plainpages=true,citecolor=blue,linkcolor=blue]{hyperref}

\marginsize{2cm}{2cm}{2cm}{2cm}
\pagestyle{fancy}
\fancyhf{Líneas de transmisión y antenas}
\fancyhead[L]{\footnotesize UPIITA-IPN} 
\fancyhead[R]{\footnotesize 3TV1} 
\fancyfoot[R]{\footnotesize Práctica 1}
\fancyfoot[C]{\thepage}
\fancyfoot[L]{\footnotesize Carta de Smith} 

\renewcommand{\footrulewidth}{0.4pt}
\renewcommand{\spanishtablename}{Tabla}
\renewcommand{\labelitemii}{$\star$}

\begin{document}
\newpage
\tableofcontents
\listoffigures
\listoftables

\newpage
\section{Balun}
Un balun es un dispositivo que une una línea balanceada (una que tiene dos conductores, 
con corrientes iguales en direcciones opuestas, como un cable de par trenzado) a una 
línea no balanceada (una que tiene sólo un conductor y una tierra, como un cable coaxial). 
Un balun es un tipo de transformador: se utiliza para convertir una señal no balanceada 
en balanceada o viceversa. Los baluns aíslan una línea de transmisión y proporcionan una 
salida balanceada. Un uso típico para un balun es en una antena de televisión. El término 
se deriva de la combinación de equilibrado y desequilibrado.
\\ \\
En un balun, un par de terminales está equilibrado, es decir, las corrientes son iguales 
en magnitud y opuestas en fase. El otro par de terminales está desequilibrado; un lado 
está conectado a tierra eléctrica y el otro lleva la señal.
\\ \\
Hay dos variaciones de este dispositivo:
\begin{itemize}
    \item El unun, que transfiere la señal de una línea desequilibrada a otra.
    \item El balbal, que transfiere la señal de una línea balanceada a otra.
\end{itemize}
Para funcionar con una eficiencia óptima, un balun debe utilizarse con cargas cuyas 
impedancias presentan poca o ninguna reactancia. Tales impedancias son llamadas 
"puramente resistivas".  Como regla general, las antenas de comunicaciones bien 
diseñadas presentan cargas puramente resistivas de 50, 75 o 300 $\Omega$, aunque unas 
pocas antenas tienen impedancias resistivas más altas.
\\ \\
Algunos baluns pueden funcionar como un transformador de impedancia entre dos sistemas 
desequilibrados si hay poca o ninguna reactancia. 

\section{Aplicaciones}
Los transformadores de balun se pueden utilizar entre varias partes de un sistema de 
comunicaciones por cable o inalámbrico. La siguiente tabla denota algunas aplicaciones 
comunes.

\begin{table}[ht]
    \centering
    \begin{tabular}{|l|l|}
    \hline
    \textbf{Equilibrado} & \textbf{Desequilibrado} \\ \hline
    Receptor de televisión & Red de cable coaxial \\ \hline
    Receptor de televisión & Sistema de antena coaxial \\ \hline
    Receptor de radiodifusión FM & Sistema de antena coaxial \\ \hline
    Antena dipolo & Línea de transmisión coaxial \\ \hline
    Línea de transmisión de cable paralelo & Salida del transmisor coaxial \\ \hline
    Línea de transmisión de cable paralelo & Entrada para receptor coaxial \\ \hline
    Línea de transmisión de cable paralelo & Línea de transmisión coaxial \\ \hline
    \end{tabular}
    \caption{Aplicaciones.}
\end{table}

\section{Cálculos}
Impedancia característica de la línea.
\begin{equation}
    Z_0=200 \ \Omega
\end{equation}
Coeficiente de reflexión.
\begin{equation}
    \Gamma(0)=\frac{Z_L-Z_0}{Z_0+Z_L}
\end{equation}
%Potencia promedio.
%\begin{equation}
%    P_{prom}=||\Gamma(0)||^{2}
%\end{equation}
Perdida de retorno.
\begin{equation}
    RL=-10\log_{10} (|\Gamma(0)|^{2})
\end{equation}

\subsection{$Z_L=47 \Omega$}
\begin{equation}
    \Gamma(0)=\frac{47-200}{47+200}=-0.6194
\end{equation}
\begin{equation}
    |\Gamma(0)|^{2}=0.836
\end{equation}
\begin{equation}
    RL=-10\log_{10} (-0.6194^{2})=4.1605
\end{equation}

\subsection{$Z_L=220 \Omega$}
\begin{equation}
    \Gamma(0)=\frac{220-200}{220+200}=0.0476
\end{equation}
\begin{equation}
    |\Gamma(0)|^{2}=0.0022
\end{equation}
\begin{equation}
    RL=-10\log_{10} (0.0476^{2})=26.4478
\end{equation}

\subsection{$Z_L=270 \Omega$}
\begin{equation}
    \Gamma(0)=\frac{270-200}{270+200}=0.1489
\end{equation}
\begin{equation}
    |\Gamma(0)|^{2}=0.0221
\end{equation}
\begin{equation}
    RL=-10\log_{10} (0.1489^{2})=4.1605=16.5421
\end{equation}

\end{document}